\documentclass[../main.tex]{subfiles}
\begin{document}
\chapter{Low-Level softwarebeveiliging}
\section{Introductie}
Een \emph{implementation-level software vulnerability} is een fout in een programma die door een aanvaller kan worden misbruikt.
In dit hoofdstuk hebben we het concreet over \emph{memory corruption vulnerabilies}.
Deze zwakheden zijn enkel relevant voor onveilige talen.
Dat zijn programmeertalen die niet controleren of programma's het geheugen correct gebruiken.
Hieronder staat een voorbeeld in ANSI C in listing \ref{l:intro}. Omdat er nergens de lengte van de input argumenten wordt nagekeken kan men in geheugen schrijven waar dat niet zou mogen.
Op die manier kan er arbitraire code worden uitgevoerd (een \emph{code injection attack}).

\lstinputlisting[style=cstyle, label=l:intro]{\codeSrc/intro_example.c}

\subsection{Achtergrond informatie}
Er kan op verschillende manieren geheugen gealloceerd worden in C:
\begin{description}
	\item[Automatisch] via locale variablen in functies.
	\item[Statisch] via globale variablen.
	\item[Dynamisch] via \lstinline[style=ilcstyle]{void *malloc(size_t size);} en \lstinline[style=ilcstyle]{new}.
\end{description}
De programmeur is verantwoordelijk voor dit geheugenbeheer van het alloceren tot het de-alloceren.
Figuur \ref{f:mem_lay} geeft de \emph{process memory layout} weer van een programma.
Hierbij groeit de heap naar boven (hogere adressen) en de stack naar onder (lagere adressen).
Indien de voorbeeld code \ref{l:intro} wordt ge\"exploiteerd bevind de geinjecteerde code zich in de Stack.
\begin{figure}
\centering
\includegraphics{\imageSrc/memory_layout.png}
\caption{Typische geheugen layout met bovenaan \textbf{hoge} adressen. De stack groeit van hoge naar lage adressen en de heap van lage naar hoge adressen.}
\label{f:mem_lay}
\end{figure}

Memory management in C is zeer fout-gevoelig typische bugs zijn o.a.:
\begin{itemize}
		\item Writing past the bound of array (zoals in listing \ref{l:intro})
		\item Dangling pointers (pointers die niet naar een geldig object wijzen.)
		\item Double freeing (Twee keer \lstinline[style=ilcstyle]{free()} aanroepen zodat nog gebruikt geheughen als ongebruikt wordt gemarkeert).
		\item Memory leaks (Stukken geheugen zijn niet correct vrijgegeven.)
\end{itemize}
Om effici\"entie redenen worden deze fouten in C niet at run time gededecteert.

Om onveilige code aan te vallen moet een aanvaller hetvolgende doen:
\begin{itemize}
		\item Een bug vinden die memory safety breekt.
				\begin{itemize}
						\item Buffer overflow
						\item Dereference van dangling pointer
						\item Gebruik van onveilige API functie. Die ofwel een buffer overflowt (e.g. \lstinline[style=ilcstyle]{strcpy()} ofwel intrinsiek onveilig geimplementeerd is in assembly (e.g. \lstinline[style=ilcstyle]{printf()}
				\end{itemize}
		\item Een interessante geheugenplaats vinden om te overschrijven.
				\begin{itemize}
								\item Code adressen/pointers: Return address, Function pointer in Virtual function table, Programma specifieke fuction pointers
								\item Pointers waar de aanvaller kan controleren wat er wordt geschreven als die dereferenced raakt (= Indirect pointer overwrite): redirect de pointer naar een andere interessante locatie en schrijf dan daar een zekere waarde.
				\end{itemize}
		\item Aanvalscode in het geheugen van het process plaatsen.
\end{itemize}

\section{Aanvalsvoorbeelden}
\subsection{Stack-based buffer overflow}
De stack is een geheugengebied dat wordt gebruikt om functie-oproepen en returns at run time te beheren.
Per oproep wordt er een \emph{activation record} of \emph{stack frame} op de stack gepusht. Daarin zitten parameters, return address, lokale variablen, \ldots
Het is eenvoudig in te zien dat als er nu een lokale variable kan worden overflowt dat er interessante geheugenplaatsen ter beschikking komen.

In de onderstaande figuren leggen we een typisch Stack Based Buffer Overflow uit. Opgelet: ook hier weer veronderstellen we dat bovenaan hoge addressen staan en dus overflowt een buffer naar boven.
In figuur \ref{f:stackBO1} tonen we een normale (niet aangevallen) stack waarbij het laatste stackframe dat van functie f0 is. De instructiepointer (IP) aan het begin van f0 staat en de (Saved) Frame Pointer ((S)FP) en Stack Pointer (SP) staan aangeduid.
\begin{figure}
\centering
\includegraphics[scale=0.7]{\imageSrc/stackBO1.png}
\caption{Stack layout als de laatst opgeroepen functie f0 is.}
\label{f:stackBO1}
\end{figure}

Vervolgens in figuur \ref{f:stackBO2} wordt functie f1 opgeroepen. Een nieuw stackframe wordt op de stack gepusht. Let erop dat in dit frame het return address nu wijst naar code net onder de call naar f1. De SFP wijst naar het frame van f0 en er daarachter plaats is voorzien voor de lokale variable \lstinline[style=ilcstyle]{buffer[]}. De stackpointer wijst natuurlijk net voorbij het frame van f1.
\begin{figure}
\centering
\includegraphics[scale=0.7]{\imageSrc/stackBO2.png}
\caption{Stack layout na call van functie f1.}
\label{f:stackBO1}
\end{figure}

In deze figuur wordt de overflow aangeduid. Als er geen bounds check gebeurd kan de buffer blijven groeien en zo eerst de SFP, en dan het return address overschrijven. Door nu in het buffer uitvoerbare code te steken en het return address naar die code te doen verwijzen kan het programma volledig worden gecontroleerd.
\begin{figure}
\centering
\includegraphics[scale=0.7]{\imageSrc/stackBO3.png}
\caption{Stack layout na buffer overflow attack op de lokale variable buffer in functie f1.}
\label{f:stackBO3}
\end{figure}
Het voorbeeld uit de introductie (zie listing \ref{l:intro}) is een goed voorbeeld van een programma kwetsbaar voor een buffer overflow attack.
\\\\
Code die vaak wordt gebruikt om te injecteren is het starten van een nieuwe shell.
Om dit te doen wordt er zogenaamde shellcode gebruikt.
Dit is een string van hexadecimale codes die dan worden geinterpreteerd als programma code.
Hieronder vind u een voorbeeld die op Linux systemen met een intel x86 architectuur kan gebruikt worden om de \emph{sh} shell te starten.

\begin{blockquote}
	\texttt{"\textbackslash xeb\textbackslash x1f\textbackslash x5e\textbackslash x89\textbackslash x76\textbackslash x08\textbackslash x31\textbackslash xc0\textbackslash x88\textbackslash x46\textbackslash x07\textbackslash x89\textbackslash x46\textbackslash x0c\textbackslash xb0\textbackslash x0b \textbackslash x89\textbackslash xf3\textbackslash x8d\textbackslash x4e \\\textbackslash x08\textbackslash x8d\textbackslash x56\textbackslash x0c\textbackslash xcd \textbackslash x80\textbackslash x31\textbackslash xdb\textbackslash x89\textbackslash xd8\textbackslash x40\textbackslash xcd\textbackslash x80\textbackslash xe8\textbackslash xdc\textbackslash xff\textbackslash xff\textbackslash xffbin/sh"}
\end{blockquote}

Om een stack based buffer overflow werkend te krijgen zijn er heel veel details waar de aanvaller rekening mee moet houden.

\subsection{Heap-based buffer overflow}
Soms bevatten programma's enkel een overflowbare buffer die op de heap gealloceerd is.
Bijvoorbeeld globaal gedeclareerde structuren.
Aangezien die zich niet op de stack bevinden is er geen return address in de nabijheid.
Wel zijn er andere manieren om toch een succesvolle code injection aanval te doen waarvan we er hier twee bekijken: overschrijven van een function pointer en het overschrijven van heap metadata.
\subsubsection{Overwriting function pointer}
Listing \ref{l:heapBO} toont een programma dat kwetsbaar is voor een heap based buffer overflow door middel van een function pointer te overschrijven.
\lstinputlisting[style=cstyle, label=l:heapBO]{\codeSrc/heapBO.c}

Figuur \ref{f:heapBO1} toont hoe de vulnerability wordt uitgebuit.
In (a) zien we de opeenvolgende adressen van de struct en de waardes die ze initieel krijgen toegekend.
In (b) zien we hoe er een overflow kan worden gerealiseerd die cmp overschrijft.

Let hierbij op het feit dat de waardes in \textit{Little Endian} worden opgeslagen.
Dit betekend dat de eerste byte van een geheugenlocatie de meest rechtse positie krijgt.
We kunnen dit makkelijk zien door het einde van het voorbeeld (b) te bekijken. De laatste drie characters van de string zijn ``sdf''.
We weten dat ``s'' overeenkomt met het hexadecimale ``73'', d met ``64'' en f met ``66''.

Als we nu naar de geheugen locatie van cmp (\emph{0x00353078}) kijken dan vinden we daar \emph{0x00666473}.
Tabel
\begin{table}
\centering
\begin{tabular}{l|cccc|}
		geheugenplaats (hex) & 0x0035307b & 0x0035307a & 0x00353079 & 0x00353078 \\ \hline
		Geheugen waarde (hex) & 00 & 66 & 64 & 73 \\ \hline
		Character (string representatie) & n/a & f & d & s \\ \hline
\end{tabular}
\end{table}

\begin{figure}
\centering
\includegraphics[scale=0.7]{\imageSrc/heapBO1.png}
\caption{Stack layout na buffer overflow attack op de lokale variable buffer in functie f1.}
\label{f:heapBO1}
\end{figure}

\subsubsection{Overwriting heap metadata}
De heap wordt gebruikt om dynamisch gealloceerde data op te slaam.
Met functies zoals \lstinline[style=ilcstyle]{malloc()} worden er dynamisch geheugenblokken gealloceerd en gedealloceerd met \lstinline[style=ilcstyle]{free()}.
De meest memory allocation libraries onthouden metadata voor of achter gebruikte blokken.
Een gevolg daarvan is dat buffer overruns op de heap deze management informatie kunnen overschrijven.
Dit maakt een \emph{indirect pointer overwrite} aanval mogelijk.

Om de aanval te begrijpen kijken we eerst naar hoe de heap er uit ziet in het normale geval.
In figuur \ref{f:heapBO2} vinden we hoe een gebruikt blok en een leeg blok eruit zien.
De lege blokken worden onthouden door ze als dubbel gelinkte lijst te gebruiken.
Achteraan elk blok wordt een backward pointer, forward pointer en wat andere informatie (groote bvb) bijgehouden.
Als een blok gebruikt wordt dan wordt het uit de lijst gehaald door de backward pointer twaalf plaatsen verder te schrijven dan naar waar de forward pointer wijst (en omgekeerd voor de forward pointer).

\begin{figure}
\centering
\includegraphics[scale=0.6]{\imageSrc/heapBO2.png}
\caption{Normaal geval in de heap als het lege blok c wordt gebruikt en dus unlinked wordt van de double linked list met vrije blokken.}
\label{f:heapBO2}
\end{figure}

Het idee is nu om dit feit uit te buiten door het unlink mechanisme te gebruiken om het return address van een functie op de stack te vervangen.
In figure \ref{f:heapBO3} wordt er in blok d een buffer geoverflowt zodat de forward en backward pointer van c kunnen overschreven worden.
Concreet gebeurdt dat door de backward pointer (groen) te doen wijzen naar in de buffer waar de geinjecteerde code staat.
De forward pointer van c moet dan wijzen naar twaalf plaatsen \emph{onder} (er wordt immers bij het unlinken twaalf plaatsen hoger geschreven) het return address (RA) van een functie.
Het eigenlijke resultaat is dat te bezichtigen in \ref{f:heapBO4}.
Van zodra blok c wordt gebruikt zal de aanval opgezet zijn.
Het return address van een zekere functie zal dan immers wijzen naar de geinjecteerde code.
Van zodra die functie dan returnt zal bijvoorbeeld de shell code worden uitgevoerd.

\begin{figure}
\centering
\includegraphics[scale=0.6]{\imageSrc/heapBO3.png}
\caption{Buffer overflow op de heap in actie. Een buffer in blok d wordt overflowt (naar boven in deze figuur) tot in het lege blok c.}
\label{f:heapBO3}
\end{figure}

\begin{figure}
\centering
\includegraphics[scale=0.6]{\imageSrc/heapBO4.png}
\caption{Resultaat van een indirect pointer overwrite door middel van een buffer overflow op de heap.}
\label{f:heapBO4}
\end{figure}

Deze aanval wordt ook wel een indirect pointer overwrite genoemd.
De aanval is breed inzetbaar in verschillende situaties.
Een programma is kwetsbaar voor zo'n aanval als het drie elementen bevat:
\begin{itemize}
		\item Een bug die toelaat een pointer te overschrijven.
		\item Die pointer wordt later gedereferenced om te schrijven.
		\item En de waarde die geschreven wordt kan worden gecontroleerd door de attacker.
\end{itemize}


\begin{figure}
\centering
\includegraphics[scale=0.6]{\imageSrc/heapBO3.png}
\caption{Buffer overflow op de heap in actie. Een buffer in blok d wordt overflowt (naar boven in deze figuur) tot in het lege blok c.}
\label{f:heapBO3}
\end{figure}

\begin{figure}
\centering
\includegraphics[scale=0.6]{\imageSrc/heapBO4.png}
\caption{}
\label{f:heapBO4}
\end{figure}
\subsection{Return-to-libc attacks}
Tot nu toe hebben we altijd rechtstreek code kunnen injecteren in een buffer.
Dit is niet altijd mogelijk aangezien er een countermeasures kunnen genomen worden om dit te voorkomen.
Wel bestaan er dan Indirect code injection attacks die het programma sturen door het manipuleren van de stack.
Dit maakt het mogelijk om stukken code uit te voeren die al in het geheugen zitten en meestal is er zo'n code beschikbaar bvb. libc.

Figuren \ref{f:libc1} en \ref{f:libc2} tonen hoe het returnen uit een functie de stack beinvloed.
\begin{figure}
\centering
\includegraphics[scale=0.8]{\imageSrc/libc1.png}
\caption{Stack als functie f2 gaat returnen. De Stack Pointer (SP) en Instruction Pointer (IP) zijn aangeduid.}
\label{f:libc1}
\end{figure}

\begin{figure}
\centering
\includegraphics[scale=0.8]{\imageSrc/libc2.png}
\caption{Stack als functie f2 gereturned heeft en enkel nog het f1 stackframe op de stack staat.}
\label{f:libc2}
\end{figure}

Aangezien we de werking van de stack begrijpen kunnen we dit gebruiken om de stack na te bootsen.
Een nagebootste stack kan gewoon in een buffer wordten opgeslagen.
De fake stack kan bijvoorbeeld een functie aanroepen die een shell start of het mogelijk maakt om andere code rechtstreeks te injecteren.
Vervoglens moeten we de SP doen wijzen naar de fake stack net voor de return instructie van de actieve functie wordt uitgevoerd.
Hiervoor wordt er vaak eerst naar trampoline code gesprongen die dan de stackpointer van locatie kan veranderen afhankelijk van het geinjecteerde address.
We bekijken het geval uit listing \ref{l:libc}. 
In figuur \ref{f:libc3} vinden we de assembler code van de qsort functie.
Let er op dat het register \texttt{ebx} net voor de call op de stack wordt gepusht en dat daarin het address van waar we de nieuwe stack gezet hebben zit.
Als de call instructie wordt uitgevoerd wordt er naar locatie \textit{0x7c971649} gesprongen. 
Op locatie \textit{0x7c971649} vinden we de assembler code in figuur \ref{lib4}.
Dit is de trampoline code die gebruikt wordt om de stack pointer te verplaatsen.
Deze code bevind zich ergens in het geheugen en komt hier uitstekend van pas om de libc attack uit te voeren. 
Aangezien in register \texttt{ebx} nog het gewilde address zit wordt de stack pointer (\texttt{esp}) nu verplaatst naar \texttt{ebx} wat dus in ons geval de tmp buffer betekent.
Verder wordt er nu een return (\texttt{ret}) gedaan en dus begint het stack-unwinden.
Op deze valse stack kan nu elke mogelijke functie met gewilde argumenten worden gezet.
\begin{figure}
\centering
\includegraphics[scale=0.8]{\imageSrc/libc3.png}
\caption{Gedeeltelijke assembler voorstelling van de qsort functie. Het register \texttt{ebx} geven we als waarde de start van de tmp buffer. En \texttt{comp\_fp} moet de waarde \textit{0x7c971649} hebben zodat er naar de trampoline code wordt gesprongen.}
\label{f:libc3}
\end{figure}

\begin{figure}
\centering
\includegraphics[scale=0.8]{\imageSrc/libc4.png}
\caption{De trampoline code die gebruikt wordt om de stack pointer naar het gewilde address te verplaatsen en dan een eerste keer te returnen. Deze code bevint zich ergens in het geheugen en als het address geweten is kan die dus worden misbruikt.}
\label{f:libc4}
\end{figure}

\begin{figure}
\centering
\includegraphics[scale=0.8]{\imageSrc/libc5.png}
\caption{De stack zoals die eruit ziet in de median functie.}
\label{f:libc5}
\end{figure}

\textit{Meer uitleg volgt mogelijk nog}

\lstinputlisting[style=cstyle, label=l:libc]{\codeSrc/libc.c}

\subsection{Data-only attacks}
Dit soort aanvallen bestaat erin enkel de data van een programma te veranderen.
Afhankelijk van die data kan dit tot interessante exploits leiden.
We bekijken er hier twee: Unix password attack en overwriting the environment table.
\subsubsection{Unix password attack}
In figuur \ref{f:UnixPA} staat een voorbeeld van een programma dat op een onveilige manier een wachtwoord controleert.
In een oude versie van Unix werd dit gebruikt maar dit is eenvoudig te omzeilen.
Als er immers een wachtwoord wordt ingegeven dat bestaat uit \emph{pw || hash(pw)} en het pw is lang genoeg dan overflowt het wachtwoord in de hash.
Bijgevolg zal er altijd toegang worden verschaft.

\begin{figure}
\centering
\includegraphics[scale=0.8]{\imageSrc/unixPA.png}
\caption{Voorbeeld van onveilig password checking programma.}
\label{f:unixPA}
\end{figure}

\subsubsection{Overwriting the environment table}
Doordat er soms assumpties worden gemaakt over de environment table door de programmeur kan dit worden uitgebuit.
In figuur \ref{f:dataonly} zien we kwetsbare code.
De waarde achter de environment variable ``SAFECOMMAND'' wordt met een system call uitgevoerd.
Maar aangezien we de offset van de data buffer en zijn waarde kunnen bepalen kunnen we in een (bijna) arbitraire geheugenlocatie schrijven en dus de waarde horend bij ``SAFECOMMAND'' veranderen in value.

\lstinputlisting[style=cstyle, label=l:dataonly]{\codeSrc/dataonly.c}

\section{Verdedigings voorbeelden}
\subsection{Stack canaries}
Het idee van een stack canary is vrij eenvoudig en tegelijk behoorlijk effectief.
Net voor elke base pointer of return address wordt een zeker waarde in het stack frame geplaatst.
Elke keer er nu een return gebeurdt wordt de waarde gecontroleerd of die al dan niet verandered is.
Die waarde noemen we dan een ``canary'', genoemd naar de canaries die werden gebruikt in de koolmijnen om giftige gassen vroegtijdig te ontdekken.
In figuur \ref{f:stackcanary} wordt er getoond hoe een overflow met een Canary eenvoudig kan worden gededecteert.

\begin{figure}
\centering
\includegraphics[scale=0.8]{\imageSrc/stackcanary.png}
\caption{Stack layout met canary als er een overflow gebeurdt is.}
\label{f:stackcanary}
\end{figure}

\subsection{Non-executable data}
Direct code injection aanvallen voeren een bepaald stuk code uit dat op de stack staat.
De meeste normale programmas doen dit echter zelden dus een countermeasure kan erin bestaan om de heap en stack volledig als non-executable data te markeren.
Dit voorkmont dan direct code injection maar geen data-only attacks of return-into-libc.
Bovendien kan het zijn dat bepaalde legacy programmas hier \textbf{wel} op steunen en dus met deze maatregel niet meer werken.

\subsection{Control-flow integrity}
De meeste aanvallen die we bespraken breken de control flow zoals die gecodeerd is in de broncode.
Een mogelijke countermeasure is dus het controleren dat de control-flow gezond blijft at runtime.
Dit kan er bijvoorbeeld in bestaan om te controleren dat een functie altijd terugspringt naar van waar hij is opgeroepen.
Andere mogelijkheden kunnen erin bestaan expliciet te controleren of pointers een geldige mogelijke waarde hebben.
Dit kan eenderzijds expliciet in de programma broncode worden gedaan.

Anderzijds kan er een control-flow graph worden opgezet die gelinkt is aan mogelijke labels naar waar een call kan springen of kan van returnen.
Met deze graph kan er dan dynamisch code geinserteerd worden die controleert of elke sprong geldig is door de labels na te kijken.
Een voorbeeld daarvan is te vinden in figuur \ref{f:cfi}.

\begin{figure}
\centering
\includegraphics[scale=0.8]{\imageSrc/cfi.png}
\caption{Afbeelding die Control Flow Integrety met labels illustreert.}
\label{f:cfi}
\end{figure}

\subsection{Layout randomization}
De meeste van deze low-level aanvallen steunen op kennis van run time memory adressen.
Door artificiele variatie in deze adressen te introduceren verhoogt de moeilijkheid voor deze aanvallen aanzienlijk.
Zo'n Adress Space Layout Randomization (ASLR) is een goedkope en effectieve countermeasure tegen dergerlijke aanvallen.

\section{Conclusie}
Door het implementeren van de countermeasures is er een soort arms-race ontstaan tussen aanvallers en verdedigers.
Countermeasures zijn zelden perfect en de aanvallers bedenken nieuwe manieren om die te omzeilen.
Hoewel het met de countermeasures veel moeilijker is om in C-family talen dergerlijke aanvallen te doen is het van security standpunt veiliger om over te stappen naar safe languages (bvb. Java of C\#).
Een samenvatting van hoe de countermeasures die we bespraken de aanvallen (trachten) te voorkomen vind u in tabel \ref{t:overview}.
\begin{table}
		\begin{tabular}{|p{3cm}| p{3cm} p{3cm} p{3cm} p{3cm}|}
				\hline
				& \textbf{Return address corruption} & \textbf{Heap Function pointer corruption} & \textbf{Jump-to-libc} & \textbf{Non-control data} \\ \hline \hline
				\textbf{Stack Canary} & Partial defense & n/a & Partial defense & Partial defense \\ \hline
				\textbf{Control-flow integrety} & Partial defense & Partial defense & Partial defense & \\ \hline
				\textbf{Control-flow integrety} &  Partial defense & Partial defense & Partial defense & \\ \hline
				\textbf{Address Space Layout Randomization} &  Partial defense & Partial defense & Partial defense & Partial defense \\ \hline
		\end{tabular}
		\caption{Overzicht van aanvallen en verdedigings mechanismes voor low-level attacks.}
		\label{t:overview}
\end{table}

De besproken ``automatische'' verdedigingsmechanismes zijn maar een deel van het beveiligen van C software.
Andere onderdelen zijn o.a.: Threat modeling, Code Review en Security Testing.
\end{document}
